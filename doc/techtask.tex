\documentclass[techtask]{espd}
\usepackage[russian]{babel}

\bibliographystyle{gost2008}

\managerrank{Доцент департамента программной инженерии, канд. техн. наук}
\manager{С.Л. Макаров}

\authorrank{студент группы БПИ183}
\author{Д.А. Молдавский}

\title{Программная среда для записи\\математических лекций}
\code{RU.17701729-04.03}

\city{Москва}
\year{2019}

\begin{document}

\annotation

\tableofcontents

\section{Введение}
\subsection{Наименование программы}
Программная среда для записи математических лекций <<LexType>>.

\subsection{Краткая характеристика области применения}
Основной областью применения данной программы является запись лекций в университетах и колледжах по математическим и техническим предметам, то есть по предметам, конспектирование которых отличается необходимостью частой записи математических формул и схематических рисунков. Также приложения может использоваться и в других сферах, не требующих возможности быстройзаписи: подготовка курсовых и дипломных работ, написание научных статей, ведение рабочих записей.

К характеристике записи лекций в компьютерном виде относятся следующие факторы:

\begin{enumerate}
\item Необходимость записывать большой объем информации в сжатое время
\item Невозможность использование компьютерной мыши для основных операций
\item Требование к краткости синтаксиса целевого языка
\item Наличие специальных команд для помощи вставки тектса
\end{enumerate}

При подготовке текста в других условиях нет необходимости записывать текст со скоростью лектора, однако возможность делать записи быстрее может сказаться на удобстве и скорости появления усталости.

\section{Основания для разработки}
\subsection{Документ, на основании которого ведется разработка}
Разработка веделтся на основании будущего приказа об утверждении тем курсовых работ в 2019-2020 учебном году.

\subsection{Наименование темы разработки}
\paragraph{Наименование темы разработки на русском языке}
Программная среда для записи математических лекций.

\paragraph{Наименование темы разработки на английском языке}
Software Environment for Mathematical Lectures Recording.

\section{Назначение разработки}
\subsection{Функциональное назначение программы}
Программа предназначена для записи текста, математических формул и графиков с использованием мышки и клавиатуры в текстовом виде в промежуточный язык, а также для перевода этого текста в формат документов.

\subsection{Эксплуатационное назначение программы}
Программа предназначена для эксплуатации студентами технических вузов, школьниками, а также сотрудников научных и образовательных учреждений, в область деятельности которых входит работа с математическим текстом.

\section{Требования к программе}
\subsection{Требования к функциональным характеристикам}
\subsection{Требования к надежности}
\subsection{Условия эксплуатации}
\subsection{Требования к составу и параметрам технических средств}
\subsection{Требования к информационной и программной совместимости}
\subsection{Требования к маркировке и упаковке}
\subsection{Требования к транспортированию и хранению}
\subsection{Специальные требования}

\section{Требования к программной документации}
\section{Технико-экономические показатели}
\section{Стадии и этапы разработки}
\section{Порядок контроля и приемки}

%\bibliography{espd}

\end{document}
