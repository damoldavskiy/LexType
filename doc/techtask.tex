\documentclass[techtask]{espd}
\usepackage[russian]{babel}
\usepackage{multirow}
\usepackage{array}

\bibliographystyle{gost2008}

\managerrank{Научный руководитель,\\доцент департамента\\программной инженерии\\факультета компьютерных наук,\\канд. техн. наук}
\manager{С.Л. Макаров}

\authorrank{студент группы БПИ183}
\author{Д.А. Молдавский}

\title{Программная среда для записи\\математических лекций}
\code{04.03}

\city{Москва}
\year{2020}

\begin{document}

\annotation
Техническое задание -- это основной документ, оговаривающий набор требований и порядок создания программного продукта, в соответствии с которым производится разработка программы, ее тестирование и приемка.

Настоящее Техническое задание на разработку <<Программной среды для записи математических лекций>> содержит следующие разделы: <<Введение>>, <<Основания для разработки>>, <<Назначение разработки>>, <<Требования к программе>>, <<Требования к программной документации>>, <<Технико-экономические показатели>>, <<Стадии и этапы разработки>>, <<Порядок контроля и приемки>>.

В разделе <<Введение>> указаны нименование программы и краткая характеристика области применения разработки.

В разделе <<Основания для разработки>> указан документ, на основании которого ведется разработка, а также наименование темы разработки.

В разделе <<Назначение разработки>> указано функциональное и эксплуатационное назначение программного продукта.

В разделе <<Требования к программе>> указаны требования по функционалу и надежности к разрабатываемому продукту, а также условия, накладываемые на технические и информационные средства, в условиях которых предполагается использование программного продукта.

В разделе <<Требования к программной документации>> указан предварительный состав программной документации.

В разделе <<Технико-экономические показатели>> указана предполагаемая потребность, а также экономические преимущества разработки по сравнению с аналогами.

В разделе <<Стадии и этапы разработки>> содержится информация о стадиях разработки и их содержании, а также сроки разработки и исполнители.

В разделе <<Порядок контроля и приемки>> указаны общие требования к приемке работы.

Настоящее Техническое задание удовлетворяет требованиям ГОСТ 19.201-79~\cite{espd201}.

\tableofcontents

\section{Введение}
\subsection{Наименование программы}
\paragraph{Наименование программы на русском языке}
Программная среда для записи математических лекций <<LexType>>.
\paragraph{Наименование программы на английском языке}
Software Environment for Mathematical Lectures Recording <<LexType>>.

\subsection{Краткая характеристика области применения}
Основной областью применения данной программы является запись лекций в университетах и колледжах по математическим и техническим предметам, то есть по предметам, конспектирование которых отличается необходимостью частой записи математических формул и схематических рисунков. Также приложение может использоваться и в других сферах, не требующих возможности быстрой записи: подготовка курсовых и дипломных работ, написание научных статей, ведение рабочих записей.

К характеристике записи лекций в компьютерном виде относятся следующие факторы:

\begin{enumerate}
\item Необходимость записывать большой объем информации в сжатое время;
\item Невозможность использование компьютерной мыши для основных операций;
\item Требование к краткости синтаксиса целевого языка;
\item Наличие специальных команд для помощи вставки тектса.
\end{enumerate}

При подготовке текста в других условиях нет необходимости записывать текст со скоростью лектора, однако возможность делать записи быстрее может сказаться на удобстве и скорости появления усталости.

\section{Основания для разработки}
\subsection{Документ, на основании которого ведется разработка}
Разработка ведется на основании приказа декана факультета компьютерных наук И.В. Аржанцева "Об утверждении тем, руководителей курсовых работ студентов образовательной программы «Программная инженерия» факультета компьютерных наук" № 2.3-02/1112-04 от 11.12.2019.

\subsection{Наименование темы разработки}
\paragraph{Наименование темы разработки на русском языке}
Программная среда для записи математических лекций.

\paragraph{Наименование темы разработки на английском языке}
Software Environment for Mathematical Lectures Recording.

\section{Назначение разработки}
\subsection{Функциональное назначение программы}
Программа предназначена для записи текста, математических формул и графиков с использованием мышки и клавиатуры в текстовом виде в промежуточный язык, перевода этого текста в язык LaTeX и дальнейшей компиляции в PDF, т.е. перевода этого текста в формат документов.

Запись текста подразумевает наличие текстового редактора и дополнительных функций (сниппетов) для возможности набирания часто повторяющихся блоков автоматически. Редактор изображений служит для создания изображений для промежуточного текста в векторном формате.

\subsection{Эксплуатационное назначение программы}
Программа предназначена для эксплуатации студентами технических вузов, школьниками, а также сотрудников научных и образовательных учреждений, в область деятельности которых входит работа с математическим текстом.

\section{Требования к программе}
\subsection{Требования к функциональным характеристикам}
\paragraph{Требования к составу выполняемых функций}
К программе относятся следующие требования:

\begin{enumerate}
\item Возможность печатать текст в препроцессорный язык (т.е. наличие текстового редактора);
\item Подсветка синтаксиса с выделением команд и макросов LaTeX;
\item Наличие возможности настраивать сниппеты (т.е. краткие команды для вставки блока текста);
\item Возможность менять цвета для подсветки и окна приложения;
\item Возможность компилировать препроцессорный язык в формат PDF;
\item Отображение скомпилированного файла PDF и его обновление при повторной компиляции;
\item Наличие настроек редактора: подсветка текущей строки, отображение нумерации строк, включение сниппетов;
\item Наличие стандартных операций текстового редактора: <<найти>>, <<заменить>>, <<вырезать>>, <<копировать>>, <<вставить>>;
\item Наличие механизма сохранения и загрузки файла (пункты меню <<открыть>>, <<сохранить>>). Сохранение должно происходить в текстовом формате;
\item Удовлетворительная производительность текстового редактора (редактирование файлов конспектов размером около нескольких килобайт не должно вызывать видимых задержек на компьютере с системными требованиями, указанными в подразделе~\ref{subsection:requirements});
\item Наличие встроенного графического векторного редактора с возможностью добавления различных векторных элементов (прямая, кривая, прямоугольник, эллипс, текст);
\item Графический редактор должен позволять удалять добавленные элементы;
\item Возможность вставки графических изображений в текст текущего файла.
\end{enumerate}

\paragraph{Требования к интерфейсу}

\begin{enumerate}
\item Наличие текстового редактора и панели инструментов на главном экране приложения;
\item Наличие номеров строк в текстовом редакторе;
\item Наличие подсветки ключевых слов и математического режима LaTeX;
\item Наличие понели инструментов в графическом редакторе;
\item Возможность менять цвета интерфейса (цвет фона и текста).
\end{enumerate}

\paragraph{Требования к организации входных данных}
Программа должна позволять импортировать ранее сохраненные записи из файловой системы.

\paragraph{Требования к организации выходных данных}
Программа должна позволять сохранять введенные пользователем данные в файловой системе. Все данные (в том числе векторные иллюстрации) должны сохраняться в текстовом формате.

\subsection{Требования к надежности}
Программа должна проверять входные данные и корректно работать при любых сценариях пользователя. Для обеспечения надежности параллельно с разработкой должны составляться юнит-тесты.

\subsection{Условия эксплуатации}
Климатические условия эксплуатации программы определяются климатическими условиями эксплуатации оборудования, используемого для хранения и запуска программы. Таким образом, должны выполняться следующие условия:

\begin{enumerate}
\item Влажность -- не более 80\%;
\item Температура -- от 10 $^\circ$C до 30 $^\circ$C;
\item Атмосферное давление от 630 до 800 мм рт. ст.;
\item Отсутствие газообразных кислот и коррозийных веществ в воздухе;
\item Запыленность не более 0.75 мг/м$^3$.
\end{enumerate}

\subsection{Требования к составу и параметрам технических средств}\label{subsection:requirements}
Для работы программы необходим компьютер, характеристики которого обеспечивают устойчивую работу установленной операционной системы. Минимальные системные требования:

\begin{enumerate}
\item Процессор с тактовой частотой 1 ГГц;
\item Оперативная память объемом 2 Гб;
\item Свободное место на жестком диске 1 Гб;
\item Наличие компьютерной мыши или сенсорной панели;
\item Наличие клавиатуры;
\item Наличие монитора.
\end{enumerate}

\subsection{Требования к информационной и программной совместимости}
Для работы программы необходима одна из следующих операционных систем:

\begin{enumerate}
\item Любые на базе ядра Linux (Debian, Ubuntu, Mint и др.);
\item Windows.
\end{enumerate}

Также необходимо предустановленное окружение LaTeX, компилятор pdflatex должен быть добавлен в PATH.

\paragraph{Для пользователя Windows}
На компьютере должна быть установлена операционная система Windows 10, в этом случае необходимо запускать .exe файл. Дополнительные библиотеки не требуется, так как библиотеки Qt и Poppler идет в составе программы.

\paragraph{Для пользователя Linux}
На компьютере должна быть установлена операционная система Ubuntu 18.04 LTS (или более поздняя). Также необходимо убедиться в том, что из официального репозитория установлены пакеты libqt5core5a, libqt5gui5, libpoppler-qt5-1.

\subsection{Требования к маркировке и упаковке}
Программа поставляется в виде архива с исполняемым файлов для целевой платформы.

\subsection{Требования к транспортированию и хранению}
Требования к транспортировке и хранению программных документов являются стандартными и должны соответствовать общим требованиям хранения и транспортировки печатной продукции:

\begin{enumerate}
\item В помещении для хранения печатной продукции допустимы температура воздуха от 10 $^\circ$С до 30 $^\circ$С и относительная влажность воздуха от 30\% до 60\%;
\item Документацию хранят и используют на расстоянии не менее 0.5 м от источников тепла и влаги. Не допускается хранение печатной продукции в помещениях, где находятся агрессивные агенты – растворители, спирт, бензин;
\item Не допускается попадание на документацию агрессивных агентов;
\item Транспортировка производится в специальных контейнерах с применением мер по предотвращению деформации документов внутри контейнеров, а также проникновения влаги, вредных газов, пыли, солнечных лучей и образованию конденсата внутри контейнеров;
\item Программные документы, предоставляемые в печатном виде, должны соответствовать общим правилам учета и хранения программных документов, предусмотренных стандартами Единой системы программной документации и соответствовать требованиям ГОСТ 19.602-78~\cite{espd602}.
\end{enumerate}

\section{Требования к программной документации}
\subsection{Предварительный состав программной документации}\label{subsection:documentation}
<<Программная среда для записи математических лекций>>. Техническое задание (ГОСТ 19.201-78~\cite{espd201})

<<Программная среда для записи математических лекций>>. Программа и методика испытаний (ГОСТ 19.301-78~\cite{espd301})

<<Программная среда для записи математических лекций>>. Пояснительная записка (ГОСТ 19.404-79~\cite{espd404})

<<Программная среда для записи математических лекций>>. Руководство оператора (ГОСТ 19.505-79~\cite{espd505})

<<Программная среда для записи математических лекций>>. Текст программы (ГОСТ 19.401-78~\cite{espd401})

\subsection{Специальные требования к программной документации}\label{subsection:docspec}
Документы к программе должны быть выполненны в соответствии с ГОСТ 19.106-78~\cite{espd106} и ГОСТами к каждому виду документа (см. п.~\ref{subsection:documentation}).

Пояснительная записка должна быть загружена в систему <<Антиплагиат>> через LMS НИУ ВШЭ.

Документация и программа сдаются в электронном виде в формате .pdf или .docx в архиве формата .zip или .rar.

За один день до защиты комиссии все материалы курсового проекта:
\begin{enumerate}
\item Техническая документация;
\item Программный проект;
\item Исполняемый файл;
\item Отзыв руководителя;
\item Лист Антиплагиата.
\end{enumerate}
должны быть загружены одним или несколькими архивами в проект дисциплины <<Курсовой проект 2019-2020>> в личном кабинете информационной образовательной среде LMS (Learning Management System) НИУ ВШЭ.

\section{Технико-экономические показатели}
\subsection{Предполагаемая потребность}
Предполагается, что программа будет использована для записи лекций по математическим предметам (линейная алгебра, математический анализ и т.д.) студентами технических вузов, а также для ведения записей со значительным количеством вставок формул и векторной графики.

\subsection{Экономические преимущества разработки по сравнению с отечественными и зарубежными аналогами}
Отличительной особенностью данного редактора от существующих аналогов (TeXstudio~\cite{texstudio}, TeXworks~\cite{texworks}) является наличие следующих функций:

\begin{enumerate}
\item Встроенный редактор графики;
\item Наличие препроцессорного языка;
\item Наличие механизма сниппетов.
\end{enumerate}

\section{Стадии и этапы разработки}
\subsection{Необходимые стадии разработки, этапы и содержание работ}

\noindent\begin{tabular}{|>{\raggedright}p{50mm}|>{\raggedright}p{55mm}|>{\raggedright\arraybackslash}p{60mm}|}
\hline
Стадии & Этапы работ & Содержание работ \\ \hline
\multirow[t]{8}{=}{1. Техническое задание} & \multirow[t]{2}{=}{Обоснование необходимости разработки программы} & Постановка исходных материалов \\ \cline{3-3}
& & Сбор исходных материалов \\ \cline{2-3}
& \multirow[t]{3}{=}{Научно-исследовательские работы} & Определение структуры входных и выходных данных \\ \cline{3-3}
& & Определение требований к техническим средствам \\ \cline{3-3}
& & Обоснование принципиальной возможности решения поставленной задачи \\ \cline{2-3}
& \multirow[t]{3}{=}{Разработка и утверждение технического задания} & Определение требований к программе \\ \cline{3-3}
& & Определение стадий, этапов и сроков разработки программы и документации на нее \\ \cline{3-3}
& & Согласование и утверждение технического задания \\ \hline
\multirow[t]{3}{=}{2. Рабочий проект} & Разработка программы & Программирование и отладка программы \\ \cline{2-3}
& Разработка программной документации & Разработка программных документов в соответствии с требованиями ГОСТ 19.101-77~\cite{espd101} \\ \cline{2-3}
& Испытания программы & Разработка, согласование и утверждение порядка и методики испытаний \\ \hline
\multirow[t]{4}{=}{3. Внедрение} & \multirow[t]{4}{=}{Подготовка и передача программы} & Утверждение даты защиты программного продукта \\ \cline{3-3}
& & Подготовка программы и программной документации для презентации и защиты \\ \cline{3-3}
& & Представление разработанного программного продукта руководителю и получение отзыва \\ \hline
\end{tabular}

\noindent\begin{tabular}{|>{\raggedright}p{50mm}|>{\raggedright}p{55mm}|>{\raggedright\arraybackslash}p{60mm}|}
\hline
\multirow[t]{4}{=}{} & \multirow[t]{4}{=}{} & Загрузка Пояснительной записки в систему Антиплагиат через LMS НИУ ВШЭ \\ \cline{3-3}
& & Загрузка материалов курсового проекта (курсовой работы) в LMS, проект дисциплины <<Курсовой проект 2019-2020>> (см. п.~\ref{subsection:docspec}) \\ \cline{3-3}
& & Защита программного продукта (курсового проекта) комиссии \\ \hline
\end{tabular}

\subsection{Сроки разработки и исполнители}
Разработка должна закончиться к 18 мая 2020 года.

\section{Порядок контроля и приемки}
\subsection{Виды испытаний}
Проверка программного продукта, в том числе и на соответствие техническому заданию, осуществляется исполнителем вместе с заказчиком согласно <<Программе и методике испытаний>>, а также пункту~\ref{subsection:docspec}.

\subsection{Общие требования к приемке работы}
Защита выполненного проекта осуществляется комиссии, состоящей из преподавателей департамента программной инженерии, в утвержденные приказом декана ФКН сроки.

\bibliography{espd,library}

\end{document}

