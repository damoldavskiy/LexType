\documentclass[explnote]{espd}
\usepackage[russian]{babel}
\usepackage{amsmath}

\bibliographystyle{gost2008}

\managerrank{Научный руководитель,\\доцент департамента\\программной инженерии\\факультета компьютерных наук,\\канд. техн. наук}
\manager{С.Л. Макаров}

\authorrank{студент группы БПИ183}
\author{Д.А. Молдавский}

\title{Программная среда для записи\\математических лекций}
\code{04.03}

\city{Москва}
\year{2020}

\begin{document}

\annotation
Пояснительная записка -- это документ, в котором описаны используемые алгоритмы, принципы функционирования и другая информация, касающаяся работы программы.

Настоящая Пояснительная записка содержит следующие разделы: <<Введение>>, <<Назначение и область применения>>, <<Технические характеристики>>, <<Ожидаемые технико-экономические показатели>>.

В разделе <<Введение>> указано наименование программы и приведены документы, на основании которых велась разработка.

В разделе <<Назначение и область применения>> приведено назначение программы и описана характеристика области ее применения.

В разделе <<Технические характеристики>> указана постановка задачи на разработку, используемые в программе алгоритмы, принципы функционирования и методы организации входных и выходных данных.

В разделе <<Ожидаемые технико-экономические показатели>> указана предполагаемая потребность и преимущества разработки над аналогами.

Настоящая Пояснительная записка удовлетворяет требованиям ГОСТ 19.404-79~\cite{espd404}.

\tableofcontents

\section{Введение}
\subsection{Наименование программы}
\paragraph{Наименование программы на русском языке}
Программная среда для записи математических лекций <<LexType>>.
\paragraph{Наименование программы на английском языке}
Software Environment for Mathematical Lectures Recording <<LexType>>.

\subsection{Условное обозначение разработки}
Условное обозначение разработки -- <<LexType>>.

\subsection{Документ, на основании которого ведется разработка}
Разработка ведется на основании приказа декана факультета компьютерных наук И.В. Аржанцева "Об утверждении тем, руководителей курсовых работ студентов образовательной программы «Программная инженерия» факультета компьютерных наук" № 2.3-02/1112-04 от 11.12.2019.

\section{Назначение и область применения}
\subsection{Область применения программы}
Существует два основных способа записи математического текста в электронном виде:

\begin{enumerate}
\item Рукописный ввод;
\item Ввод с клавиатуры.
\end{enumerate}

Часто выбирается первый вариант, так как он наиболее прост в освоении. Однако использование рукописного ввода значительно ограничено: получающийся текст не имеет строгого форматирования и, как следствие, труднее воспринимается. По этой причине такой способ используется только для записей конспектов, не предназначенных для чтения большим числом людей. Кроме того, требуется дополнительное оборудование: планшет и стилус.

Ввод информации с клавиатуры подразумевает владение быстрой печатью и знание программного инструмента для эффективного использования. Однако при использовании клавиатурного ввода получающийся документ имеет значительно более высокое качество и более пригоден для дальнейшего использования, благодаря строгому форматированию. Также, для использования такого способа ввода достаточно иметь ноутбук.

На данный момент сущесвует два наиболее популярных способа ввода математического текста: в программе Word, а также на языке LaTeX. Word имеет ряд недостатков: платность, невозможность строгого контроля форматирования (регулярно такие элементы, как изображения, ставятся в документ <<на глазок>>), ориентированность на работу мышкой, на что уходит некоторое время при записи. LaTeX -- гибкий язык разметки, созданный для этой цели. Однако он не был рассчитан на максимальное удобства ввода, а в существующих редакторах отсутствуют некоторые важные функции, например -- вставка изображений.

\subsection{Назначение разработки}
\paragraph{Функциональное назначение программы}
Данная разработка предназначена для решения проблемы записи математического текста. В основе программы работа с языком LaTeX, дополненный препроцессором и дополнительными, нетипичными для LaTeX редактора инструментами. Таким образом, используя LexType, пользователю предоставлена возможность использовать разметку и общий стиль записи  LaTeX, но с возможностью записывать формулы быстрее, хранить их в более удобночитаемом виде и легко вставлять схематические рисунки.

\paragraph{Эксплуатационное назначение программы}
Основным образом программа предназначена для эксплуатации студентами технических вузов и колледжей. Благодаря системе сниппетов и удобном представлении формул конспектирование в компьютерном виде значительно облегчается. Сам же формат электронной записи обладает рядом преимущств перед записью ручкой -- конспект легко дублируется, резервируется и значительно проще воспринимается.

\section{Требования к программе}
\subsection{Требования к функциональным характеристикам}
\paragraph{Требования к составу выполняемых функций}
К программе относятся следующие требования:

\begin{enumerate}
\item Возможность печатать текст в препроцессорный язык (т.е. наличие текстового редактора);
\item Подсветка синтаксиса с выделением команд и макросов LaTeX;
\item Наличие возможности настраивать сниппеты (т.е. краткие команды для вставки блока текста);
\item Возможность менять цвета для подсветки и окна приложения;
\item Возможность компилировать препроцессорный язык в формат PDF;
\item Отображение скомпилированного файла PDF и его обновление при повторной компиляции;
\item Наличие настроек редактора: подсветка текущей строки, отображение нумерации строк, включение сниппетов;
\item Наличие стандартных операций текстового редактора: <<найти>>, <<заменить>>, <<вырезать>>, <<копировать>>, <<вставить>>;
\item Наличие механизма сохранения и загрузки файла (пункты меню <<открыть>>, <<сохранить>>). Сохранение должно происходить в текстовом формате;
\item Удовлетворительная производительность текстового редактора (редактирование файлов конспектов размером около нескольких килобайт не должно вызывать видимых задержек на компьютере с системными требованиями, указанными в подразделе~\ref{subsection:requirements});
\item Наличие встроенного графического векторного редактора с возможностью добавления различных векторных элементов (прямая, кривая, прямоугольник, эллипс, текст);
\item Графический редактор должен позволять удалять добавленные элементы;
\item Возможность вставки графических изображений в текст текущего файла.
\end{enumerate}

\paragraph{Требования к интерфейсу}

\begin{enumerate}
\item Наличие текстового редактора и панели инструментов на главном экране приложения;
\item Наличие номеров строк в текстовом редакторе;
\item Наличие подсветки ключевых слов и математического режима LaTeX;
\item Наличие понели инструментов в графическом редакторе;
\item Возможность менять цвета интерфейса (цвет фона и текста).
\end{enumerate}

\section{Технические характеристики}
\subsection{Постановка задачи на разработку программы}
Разрабатываемая программа должна обеспечивать возможность записи математических тектов (конспектов) с использованием настраиваемых сниппетов, с возможностью вставки схематических изображений без выхода из программы в векторном виде, а также компиляции текста. Подробное описание требований содержится в Техническом задании.

\subsection{Описание используеумых алгоритмов}
\paragraph{Алгоритмы текстового редактора}
Текстовый редактор реализован вручную, для возможности произвольной кастомизации в дальнейшем. При этом стояла задача решения нескольких задач, обоснование выбранных алгоритмов приведено далее.

Для хранения текста реализована структура данных GapBuffer~\cite{gapbuffer}.  Эта структура данных является общем случаем списка, хранящая незанятые ячейки в произвольном месте. В реализации поддерживается свободные ячейки сразу после позиции последнего удаления (добавления) текста. При этом, аналогично списку, при истощении свободных ячеек, память выделяется заново, с увеличением общей доступной памяти в 2 раза.

Данный подход выгоден при использовании в текстовом редакторе, так при редактировании текста, как правило, не используется случайный доступ, то есть большинство изменений (вставки и удаления) происходят рядом, в близких индексах. Если для списка вставка символа зависит от расстояния до конца и в худшем случае $O(N^2)$ для вставки $N$ символов, для выполнения данного худшего случая в GapBuffer необходимо, чтобы эти символы вставлялись в противоположные концы текста, что не является типичным способом использования текстового редактора. Иллюстрация такого подхода на рисунке.

\illustration[Иллюстрация GapBuffer][Как видно, вставка трех новых символов при наличии окна в 4 символа занимает $O(1)$ времени для каждого символа. Для списка необходимо было бы перемещать все символы для вставки кажого]{img/gapbuffer}

Другой решенной задачей является хранение строк. При работе с текстовым редактором необходимо иметь операцию получения свига в тексте по номеру строки $getShift(line)$, а также получение номера строки по сдвигу $getLine(shift)$. Для решения этой задачи был реализован список строк, который хранит строки как список сдвигов. При этом на на позиции $i$ стоит сдвиг $shift_i$ для строки под номером $i$. Таким образом, сложность операций:

\begin{enumerate}
\item Получение сдвига для строки -- $O(1)$;
\item Получение строки по сдвигу (используется бинарный поиск) -- $O(\log N)$, где $N$ -- количество строк;
\item Изменение текста -- $O(N)$, если изменения касаются переводов строк, $O(1)$ иначе.
\end{enumerate}

Еще одной задачей стало хранений ширины строк. В текстовом редакторе необходимо постоянно помнить максимальную ширину и высоту (для поддержки проматывания окна). Редактор не поддерживает переводы строк, поэтому высота всегда вычисляется по формуле $font\_size \cdot lines\_count$ за асимптотику $O(1)$. Для нахождения ширины реализована структура MaxVector, в которой всегда поддерживается индекс максимального элемента. При обновлении максимальной строки происходит поиск новой максимальной строки. Таким образом, сложность операций в этом алгоритме:

\begin{enumerate}
\item Изменение строки, которая не является наиболее широкой -- $O(K)$, где $K$ -- длина изменяемой строки;
\item Изменение строки, которая является наиболее широкой -- $(O(N + K)$, где $N$ -- количество строк.
\end{enumerate} 

Находить длину строки за $O(1)$ невозможно, так как ширина символов может быть разной, также существует табуляция, из-за которой при вставки символа с шириной $w$ длина строки может увеличиться в диапазоне $[0; \max(w, w_{tab})]$, где $w_{tab}$ -- это ширина табуляции.

Еще одной решенной задачей стала подсветка синтаксиса. При любом изменении в тексте происходит перерасчет разметки. Для этого создан класс MarkupModel, который для каждого символа в тексте хранит способ его подсветки. С диаграммой включений классов, которые относятся к редактору, можно ознакомиться в приложении~\ref{attachment:editor}.

Для работы сниппетов был введен соответствующий класс Snippet, содержащий информацию о его триггере, контексте и целевой строки. Класс зарегистрирован в классовой системе Qt и имеет возможность сериализации (сериализация происходит путем побайтовой записи трех его полей, аналогично происходит десериализация). Основной метод этого класса -- это метод apply, который принимает на вход объект Editor. Сниппет проверяет, возможна для его вставка (для этого он получает у редактора текущую позицию курсора) и применяет изменение. Функция возвращает флаг -- сработал ли данный сниппет.

Для удобства использования сниппетов был реализован класс SnippetManager, зарегистрированный в системе классов Qt и имеющий возможность сериализации (при сериализации сначала записывается целое 4-байтовое число -- количество сниппетов, далее сериализуются все хранимые сниппеты). Класс имеет метод reset для инициализации начального набора сниппетов и метод apply, поочередно применяющий все хранимые сниппеты к переданному экземпляру редактора. В том числе, такой подход позволяет иметь несколько наборов сниппетов для разных редакторов и управлять их содержимым раздельно.

Таким образом, при изменении текста в редакторе происходит последовательное использование следующих операций:

\begin{enumerate}
\item Изменение текста в GapBuffer;
\item Обновление переводов строк;
\item Обновление ширины строк;
\item Обновление разметки синтаксической подсветки;
\item Обновление интерфейса.
\end{enumerate}

Диаграмма вызовов в методе вставки класса Editor приведена в приложении \ref{attachment:insert}.

\paragraph{Алгоритмы редактора изображений}
Большинство вычислений, производимых в редакторе изображений, не целесообразно описывать, так как основаны на элементарной логике. Для рисования путей используется кривая Безье~\cite{bezie} (3 или 2 порядка). При движении пользователем курсора путь вычисляется как:
\begin{equation}
P[i + 1] = \begin{cases}P_u &, \text{dist}(P_u, P[i]) > \lambda \\ null &, \text{иначе}\end{cases}
\end{equation}
где $P_u$ -- это текущая точка пользователя, а $\lambda$ -- это предел, влияющий на смещение. Он задан программно и равен 30 пикселям.

Кривые рисуются через каждые 4 точки, используя 2 и 3 точку как опорные, таким образм, опорные точки двух соседних кривых и их общая точка лежат на одной прямой, и итоговая кривая получается плавной. При образовании кольца может использоваться кривая 2 порядка, чтобы возместить отсутствие одной из опорных точек.

Архитектурно была реализована структура инструментов графического редактора. Целью было минимизировать количество повторяющегося кода и упростить ввод нового функционала. Для этого в базовом классе Figure определен ряд виртуальных функций (событие нажатия кнопки, отпуска, перевод фигуры в язык TikZ, очистка, отрисовка). Также определено несколько промежуточных классов для обобщения некоторых свойств фигур (наличие контура, точек окончания и начала, возможности закраски). Диаграмма наследования всех фигур приведена на диаграмме в приложении~\ref{attachment:figure}.

\paragraph{Алгоритмы просмотрщика PDF}
В основе просмотрщика лежит библиотека Poppler~\cite{poppler}. Так как программа должна быть выполнена в единых цветах, изображение PDF файла необходимо также менять. Для этого перед отображением над изображением документа, побайтово, выполняется следующая операция:
\begin{equation}
B[i] = C_1[i\text{ mod }3] \cdot B[i] + C_2[i\text{ mod }3] \cdot (255 - B[i])
\end{equation}
где $C_1$ -- это цвет текста и $C_2$ -- это цвет фона. Цвет представляется в виде массива из трех байт. Таким образом, черно-белое изображение, которое представляет из себя документ после компиляции, может отображаться с произвольными цветами, для удобства использования темного стиля программы. 

\subsection{Описание и обоснование выбора метода организации входных и выходных данных}
\paragraph{Организация входных данных}
Входные данные для редактора -- это текст на языке LaTeX с вставками формул нового математического режима. Данный подход удобен, так как позволяет легко переводить его в язык LaTeX. Также для хранения изображений используется текстовый формат TikZ -- это также является удобным решением, так как все необходимые изображения включены в файл в эффективном как для хранения, так и для чтения формате.

\paragraph{Организация выходных данных}
При сохранении в формате .tex текст из редактора проходит через препроцессор перед сохранением, переводя новый математический формат в стандартный вид TeX. При использовании любого другого формата файл сохраняется в оригинальном виде.

При компиляции всегда в директории с файлом создается копия с дополнительным суффиксом .tex, для компилятора pdflatex подается уже этот файл. Это также является удобным, так как файл хранится таким, каким он представлен в окне редактора, при этом пользователь может ознакомиться с результатом работы препроцессора.

Также одной из составляющих языка является пакет lexpack, содержащий некоторые макросы для более удобной записи. В том числе, реализованы окружения <<Теорема>>, <<Следствие>>, <<Докаталеьство>> и другие, настроен их вид. С видом документа, в котором использованы эти макросы, можно ознакомиться в приложении~\ref{attachment:example}.

\subsection{Описание и обоснование выбора состава технических средств}
Для обеспечения функционирования программы необходим компьютер со следующими характеристиками:

\begin{enumerate}
\item Процессор с тактовой частотой 1 ГГц;
\item Оперативная память объемом 2 Гб;
\item Свободное место на жестком диске 1 Гб;
\item Наличие компьютерной мыши или сенсорной панели;
\item Наличие клавиатуры;
\item Наличие монитора.
\end{enumerate}

Такие требования обусловлены необходимостью работы целевой операционной системы и основных способов ввода информации в программе.

\subsection{Описание и обоснование выбора программных средств}
Для работы программы необходима установленная операционная система Ubuntu 18.04 LTS (или более высокая версия LTS) или Windows 10. Программа изначально предназначалась быть кроссплатформенной, для охвата большей аудитории. В том числе, это было одной из причин выбора Qt Framework~\cite{qt} для программирования интерфейса.

\section{Ожидаемые технико-экономические показатели}
\subsection{Предполагаемая потребность}
Конечная программа должна в значительной степени облегчить запись лекций в электронном формате. В связи с этим предполагается, что программа будет востребована среди студентов технических специальностей.

\subsection{Экономические преимущества разработки по сравнению с аналогами}
Уникальными особенностями данной программы является наличие встроенного векторного редактора изображений и математический режим с интеграцией Unicode-символов. Данной функции нет в редакторах TeXworks~\cite{texworks} и TeXstudio~\cite{texstudio}, наиболее популярных редакторах для LaTeX. Похожую сборку можно собрать из уже существующих приложений и утилит, однако это требует специализированных навыков пользователя, также такое решение не является кросс-платформенным.

\bibliography{espd,library}

\begin{terms}
\term{LaTeX}{набор макрорасширений для системы разметки TeX, язык для подготовки документов}
\term{Кросс-платформенность}{доступность программы для более чем одного семейства операционных систем}
\term{Сниппет}{некоторая подстрока $s_1$, заменяющаяся в тексте на другую подстроку $s_2$ при срабатывании какого-либо триггера, обычно -- нахождение каретки в конце $s_1$ при завершении операции ввода}
\end{terms}

\attachment{Справочное}{Диаграмма включений заголовочных файлов редактора}\label{attachment:editor}
\image{img/editor_diagram}

\attachment{Справочное}{Диаграмма вызовов при вставке текста}\label{attachment:insert}
\image{img/insert_diagram}

\attachment{Справочное}{Диаграмма наследования фигур}\label{attachment:figure}
\image[0.6]{img/figure_diagram}

\attachment{Справочное}{Пример документа}\label{attachment:example}
\image[0.8]{img/example}

\end{document}

