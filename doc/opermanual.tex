\documentclass[opermanual]{espd}
\usepackage[russian]{babel}

\bibliographystyle{gost2008}

\managerrank{Научный руководитель,\\доцент департамента\\программной инженерии\\факультета компьютерных наук,\\канд. техн. наук}
\manager{С.Л. Макаров}

\authorrank{студент группы БПИ183}
\author{Д.А. Молдавский}

\title{Программная среда для записи\\математических лекций}
\code{04.03}

\city{Москва}
\year{2020}

\begin{document}

\annotation

Руководство оператора -- это документ, содержащий сведения для обеспечения корректного использования программного продукта оператором.

Настоящее Руководство оператора программы <<Программная среда для записи математических лекций>>  содержит следующие разделы: <<Назначение программы>>, <<Условия выполнения программы>>, <<Выполнение программы>>, <<Сообщения оператору>>.

В разделе <<Назначение программы>> указано эксплуатационное назначение и краткая характеристика функциональности программы.

В разделе <<Условия выполнения программы>> указаны требования к аппаратным и программным средствам, используемым оператором.

В разделе <<Выполнение программы>> указана последовательность действий, включающая использование всех основных функциональных возможностей и корректному завершению программы.

В разделе <<Сообщения оператору>> указаны основные сообщения программы, которые оператор может получить при использовании.

Настоящее Руководство оператора удовлетворяет требованиям ГОСТ 19.505-79~\cite{espd505}.

\tableofcontents

\section{Назначение программы}
Программа предназначена для редактирования математического текста с возможностью вставок изображений в векторном виде и использования настраиваемых сниппетов, а также для просмотра скомпилированных в формат PDF файлов.

\section{Требования к программе}
\subsection{Требования к функциональным характеристикам}
\paragraph{Требования к составу выполняемых функций}
К программе относятся следующие требования:

\begin{enumerate}
\item Возможность печатать текст в препроцессорный язык (т.е. наличие текстового редактора);
\item Подсветка синтаксиса с выделением команд и макросов LaTeX;
\item Наличие возможности настраивать сниппеты (т.е. краткие команды для вставки блока текста);
\item Возможность менять цвета для подсветки и окна приложения;
\item Возможность компилировать препроцессорный язык в формат PDF;
\item Отображение скомпилированного файла PDF и его обновление при повторной компиляции;
\item Наличие настроек редактора: подсветка текущей строки, отображение нумерации строк, включение сниппетов;
\item Наличие стандартных операций текстового редактора: <<найти>>, <<заменить>>, <<вырезать>>, <<копировать>>, <<вставить>>;
\item Наличие механизма сохранения и загрузки файла (пункты меню <<открыть>>, <<сохранить>>). Сохранение должно происходить в текстовом формате;
\item Удовлетворительная производительность текстового редактора (редактирование файлов конспектов размером около нескольких килобайт не должно вызывать видимых задержек на компьютере с системными требованиями, указанными в подразделе~\ref{subsection:requirements});
\item Наличие встроенного графического векторного редактора с возможностью добавления различных векторных элементов (прямая, кривая, прямоугольник, эллипс, текст);
\item Графический редактор должен позволять удалять добавленные элементы;
\item Возможность вставки графических изображений в текст текущего файла.
\end{enumerate}

\paragraph{Требования к интерфейсу}

\begin{enumerate}
\item Наличие текстового редактора и панели инструментов на главном экране приложения;
\item Наличие номеров строк в текстовом редакторе;
\item Наличие подсветки ключевых слов и математического режима LaTeX;
\item Наличие понели инструментов в графическом редакторе;
\item Возможность менять цвета интерфейса (цвет фона и текста).
\end{enumerate}

\section{Условия выполнения программы}
\subsection{Требования к аппаратным средствам}
Для работы приложения компьютер должен обеспечивать корректную работу с целевой операционной системой (Windows 10 или Ubuntu 18.04) и системными программами:
\begin{enumerate}
\item Процессор с тактовой частотой не ниже 1 ГГц;
\item Объем оперативной памяти не меньше 2 Гб;
\item Свободное место на жестком диске по крайней мере 1 Гб;
\item Наличие мышки или сенсорной панели, клавиатуры, монитора.
\end{enumerate}

\subsection{Требования к программным средствам}
\paragraph{Для пользователя Windows}
На компьютере должна быть установлена операционная система Windows 10, в этом случае необходимо запускать .exe файл. Дополнительные библиотеки не требуется, так как библиотеки Qt и Poppler идет в составе программы. Необходим установленный дистрибутив LaTeX с добавленным в PATH компилятором pdflatex.

\paragraph{Для пользователя Linux}
На компьютере должна быть установлена операционная система Ubuntu 18.04 LTS (или более поздняя). Также необходимо убедиться в том, что из официального репозитория установлены пакеты texlive, libqt5core5a, libqt5gui5, libpoppler-qt5-1.

\section{Выполнение программы}
Далее описана последовательность действий, при которых будут использованы все базовые возможности программы.

Откроем программу (editor). Напишем команду doc, нажмем Tab. Напишем команду sec, нажмем Tab 2 раза и впишем текст <<Заголовок>> (\ref{section}).

\illustration[Сниппеты][Использование сниппетов базовой структуры документа]{img/section}[section]

Видно, что мы использовали три сниппета, для создания основной структуры документа, для вставки заголовка и для удаления нумерации этого заголовка (в стиле LaTeX). Теперь перейдем на новую строку и впишем следующее: <<Текст `for x exi y : x * y = 1`>>. При этом не нужно менять раскладку клавиатуры после русского текста, так как в математическом контексте раскладка меняется на противоположную, данная опция включена по умолчанию, отключить ее можно в настройках, во вкладке <<Math mode>>. Результат видно на \ref{math}.

\illustration[Вариации сниппетов][Математический контекст сниппетов]{img/math}[math]

Как видно, основная часть знаков приобрела вид символов Unicode, что значительно облегчает их восприятие. Продемонстрированный набор сниппетов является частью общего набора для облегчения конспектирования. Благодаря им можно создавать как структурные элементы (<<thm>> - теорема, <<sec>> - раздел, <<doc>> - заготовка документа), так и элементы формул (<<lam>> - лямбда, <<ge>> - знак  <<больше или равно>>). С полным списком сниппетов можно ознакомиться в настройках, пункт <<Snippets>>. Там же можно их изменить. Перейдем на следующую строку, нажмем Ctrl+D и с помощью кривых и их модификаторов нарисуем фигуру (\ref{curve}):

\illustration[Кривые][Кривые линии в режиме графического редактора]{img/curve}[curve]

Только что был открыт графический редактор. для его открытия необходимо выбрать соответствующий пункт меню или нажать Ctrl+D. В открытом редакторе для сохранения результата рисования в тексте необходимо снова нажать Ctrl+D, для отмены вставки и закрытия редактора -- Ctrl+Shift+D. Если поле пусто, любой из этих способов приведет к отмене вставки. На верхней части редактора располагается панель инструментов. Для выбора того или иного инструмента нужно нажать на него, для их различия при наведении курсора отображается название этого инструмента.

При выборе инструмента справа от пункта <<Text>> отображается доступный набор модификаторов для этого текста. Например, для эллипсов доступны настройки контура и заполнения, для линий -- настройки контура и отображения стрелок.

Для вставки текста с формулой нажмем на букву T (режим ввода текста) и введем общую формулу линейной функции, используя сниппеты и математический режим (текст -- `f(x) = alp * x + bet`) (\ref{text}):

\illustration[Текст][Вставка формул в изображение]{img/text}[text]

После проделанных операций снова нажмем Ctrl+D и вставим изображение в документ. Перейдем в начало документа нажав на область перед первым символов в тексте, нажмем Ctrl+F (или выберем пункт <<Edit - Find>>), введем текст <<pict>> и нажмем <<Find>>. Увидим результат поиска (\ref{find}):

\illustration[Поиск][Результат поиска подстроки]{img/find}[find]

Видим, что поиск работает. В данном окне также доступен флаг чувствительности к регистру. Кроме этого, доступны пункты <<Replace>> и <<Replace all>> -- соответственно, замена первой встретившейся подстроки в тексте на данный заменитель, или всех.

Сохраним файл под произвольным названием на жестком диске, нажмем F5. Далее запустим просмотрщик (программа viewer) и откроем скомпилированный в директорию с исходником PDF файл. Во время компиляции в нижней части редактора отображается список сообщений компиляции, который передает процесс pdflatex. При этом, оттянув шторку в редакторе со строки состояния вверх, можно видеть сообщения компиляции (\ref{compile}) в любой момент, в том числе, после компиляции:

\illustration[Компиляция файла][Результат компиляции документа]{img/compile}[compile]

Таким образом, весь функционал программы был продемонстрирован.

\section{Сообщения оператору}
\subsection{Виды сообщений}
Сообщения могут быть двух основных видов:
\begin{enumerate}
\item Диалоговые сообщения;
\item Статусные сообщения.
\end{enumerate}

Диалоговые сообщения имеют классический вид всплывающего окна с кнопками выбора. Однако большая возможных сообщений (статусные сообщения) имеют вид всплывающего предупреждения в строке статуса. Они имеют вид строки, отображающейся в определенный момент, которая автоматически скрывается, поэтому изображения этих сообщений не приведены. Такой вид менее функционален, но не мешает работе и имеет назначение только уведомить пользователя о некоторых действиях.

\subsection{Диалоговые сообщения}
\term{Incorrect final caret index}{отображается, когда пользователь задал в настройках сниппета некорректный индекс конечного положения каретки (не число, или выходит за рамки допустимого диапазона (\ref{caret}))}

\illustration[][][0.3]{img/caret}[caret]

\term{Changes unsaved. Do you really want to exit the program?}{отображается, когда пользователь производит попытку 
выхода из программы при не сохраненном файле. В завимисмости от ответа, выход из программы будет остановлен или программа завершится и введенные несохраненные данные будут утеряны (\ref{quit})}

\illustration[][][0.5]{img/quit}[quit]

\subsection{Статусные сообщения}
\term{Opened file: FILE}{в редакторе открыт файл FILE}
\term{Saved file: FILE}{в файловой системе сохранен файл FILE}
\term{File not opened}{открытие файла было отменено пользователем}
\term{To perform compilation, save file}{была предпринята попытка компиляции не сохраненного файла. Необходимо сохранить файл на диске}
\term{Figure insertion canceled}{Вставка фигуры была отменена пользователем}
\term{Figure inserted}{Фигура была переведена в формат TikZ и вставлена в редактор}

\bibliography{espd,library}

\end{document}
