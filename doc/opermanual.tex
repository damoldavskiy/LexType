\documentclass[opermanual]{espd}
\usepackage[russian]{babel}

\bibliographystyle{gost2008}

\managerrank{Научный руководитель,\\доцент департамента\\программной инженерии\\факультета компьютерных наук,\\канд. техн. наук}
\manager{С.Л. Макаров}

\authorrank{студент группы БПИ183}
\author{Д.А. Молдавский}

\title{Программная среда для записи\\математических лекций}
\code{04.03}

\city{Москва}
\year{2020}

\begin{document}

\annotation

Руководство оператора -- это документ, содержащий сведения для обеспечения корректного использования программного продукта оператором.

Настоящее Руководство оператора программы <<Программная среда для записи математических лекций>>  содержит следующие разделы: <<Назначение программы>>, <<Условия выполнения программы>>, <<Выполнение программы>>, <<Сообщения оператору>>.

В разделе <<Назначение программы>> указано эксплуатационное назначение и краткая характеристика функциональности программы.

В разделе <<Условия выполнения программы>> указаны требования к аппаратным и программным средствам, используемым оператором.

В разделе <<Выполнение программы>> указана последовательность действий, включающая использование всех основных функциональных возможностей и корректному завершению программы.

В разделе <<Сообщения оператору>> указаны основные сообщения программы, которые оператор может получить при использовании.

Настоящее Руководство оператора удовлетворяет требованиям ГОСТ 19.505-79~\cite{espd505}.

\tableofcontents

\section{Назначение программы}
Программа предназначена для редактирования математического текста с возможностью вставок изображений в векторном виде и использования настраиваемых сниппетов, а также для просмотра скомпилированных в формат PDF файлов.

\section{Условия выполнения программы}
\subsection{Требования к аппаратным средствам}
Для работы приложения компьютер должен обеспечивать корректную работу с целевой операционной системой (Windows 10 или Ubuntu 18.04) и системными программами:
\begin{enumerate}
\item Процессор с тактовой частотой не ниже 1 ГГц;
\item Объем оперативной памяти не меньше 2 Гб;
\item Свободное место на жестком диске по крайней мере 1 Гб;
\item Наличие мышки или сенсорной панели, клавиатуры, монитора.
\end{enumerate}

\subsection{Требования к программным средствам}
\paragraph{Для пользователя Windows}
На компьютере должна быть установлена операционная система Windows 10, в этом случае необходимо запускать .exe файл. Дополнительные библиотеки не требуется, так как библиотеки Qt и Poppler идет в составе программы.

\paragraph{Для пользователя Linux}
На компьютере должна быть установлена операционная система Ubuntu 18.04 LTS (или более поздняя). Также необходимо убедиться в том, что из официального репозитория установлены пакеты libqt5core5a, libqt5gui5, libpoppler-qt5-1.

\section{Выполнение программы}
Интерфейс программы минималистичен и понятен, поэтому у пользователя, знакомого с системой подготовки документов LaTeX, не должно возникнуть дополнительных трудностей. Далее описана последовательность действий, при которых будут использованы все базовые возможности программы.

Откроем программу. Напишем команду doc, нажмем Tab. Напишем команду sec, нажмем Tab 2 раза и впишем текст <<Заголовок>> (\ref{section}).

\illustration[Сниппеты][Использование сниппетов базовой структуры документа]{img/section}[section]

Видно, что мы использовали три сниппета, для создания основной структуры документа, для вставки заголовка и для удаления нумерации этого заголовка (в стиле LaTeX). Теперь перейдем на новую строку и впишем следующее: <<Текст `for x exi y : x * y = 1`>>. При этом не нужно менять раскладку клавиатуры после русского текста, так как в математическом контексте раскладка меняется на противоположную, данная опция включена по умолчанию. Результат видно на \ref{math}.

\illustration[Вариации сниппетов][Математический контекст сниппетов]{img/math}[math]

Как видно, основная часть знаков приобрела вид символов Unicode, что значительно облегчает их восприятие. Перейдем на следующую строку, нажмем Ctrl+D и с помощью кривых и их модификаторов нарисуем фигуру (\ref{curve}):

\illustration[Кривые][Кривые линии в режиме графического редактора]{img/curve}[curve]

После этого нажмем на букву T (режим ввода текста) и введем общую формулу линейной функции, используя сниппеты и математический режим (текст -- `f(x) = alp * x + bet`) (\ref{text}):

\illustration[Текст][Вставка формул в изображение]{img/text}[text]

После проделанных операций снова нажмем Ctrl+D и вставим изображение в документ. Перейдем в начало документа, нажмем Ctrl+F, введем текст <<pict>> и нажмем <<Find>>. Увидим результат поиска (\ref{find}):

\illustration[Поиск][Результат поиска подстроки]{img/find}[find]

Видим, что поиск работает. Сохраним файл под произвольным названием, нажмем F5 и откроем скомпилированный PDF файл с помощью поставляемого с программой просмотрщика. При этом, оттянув шторку в редакторе со строки состояния вверх, можно видеть сообщения компиляции (\ref{compile}):

\illustration[Компиляция файла][Результат компиляции документа]{img/compile}[compile]

Таким образом, весь функционал программы был продемонстрирован.

\section{Сообщения оператору}
\subsection{Виды сообщений}
Сообщения могут быть двух основных видов:
\begin{enumerate}
\item Диалоговые сообщения;
\item Статусные сообщения.
\end{enumerate}

Диалоговые сообщения имеют классический вид всплывающего окна с кнопками выбора. Однако большая возможных сообщений (статусные сообщения) имеют вид всплывающего предупреждения в строке статуса. Такой вид менее функционален, но не мешает работе и имеет назначение только уведомить пользователя о некоторых действиях.

\subsection{Диалоговые сообщения}
\term{Incorrect final caret index}{отображается, когда пользователь задал в настройках сниппета некорректный индекс конечного положения каретки (не число, или выходит за рамки допустимого диапазона)}
\term{Changes unsaved. Do you really want to exit the program?}{отображается, когда пользователь производит попытку выхода из программы при не сохраненном файле}

\subsection{Статусные сообщения}
\term{Opened file: FILE}{в редакторе открыт файл FILE}
\term{Saved file: FILE}{в файловой системе сохранен файл FILE}
\term{File not opened}{открытие файла было отменено пользователем}
\term{To perform compilation, save file}{была предпринята попытка компиляции не сохраненного файла. Необходимо сохранить файл на диске}
\term{Figure insertion canceled}{Вставка фигуры была отменена пользователем}
\term{Figure inserted}{Фигура была переведена в формат TikZ и вставлена в редактор}

\bibliography{espd,library}

\end{document}
