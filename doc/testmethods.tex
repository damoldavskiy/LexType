\documentclass[testmethods]{espd}
\usepackage[russian]{babel}

\bibliographystyle{gost2008}

\managerrank{Научный руководитель,\\доцент департамента\\программной инженерии\\факультета компьютерных наук,\\канд. техн. наук}
\manager{С.Л. Макаров}

\authorrank{студент группы БПИ183}
\author{Д.А. Молдавский}

\title{Программная среда для записи\\математических лекций}
\code{04.03}

\city{Москва}
\year{2020}

\begin{document}

\annotation

Программа и методика испытаний -- это документ, содержащий набор требований к программному продукту и программной документации, а также содержащий методику испытаний программного продукта.

Настоящая Программа и методика испытаний продукта <<Программная среда для записи математический лекций>> содержит следующие разделы: <<Объект испытаний>>, <<Цель испытаний>>, <<Требования к программе>>, <<Требования к программной документации>>, <<Средства и порядок испытаний>>, <<Методы испытаний>>.

В разделе <<Объект испытаний>> указано наименование программы, область применения и обозначение программы.

В разделе <<Цель испытаний>> указана цель испытаний программного продукта.

В разделе <<Требования к программе>> указаны требования по функционалу и надежности к разрабатываемому продукту.

В разделе <<Требования к программной документации>> указан состав программной документации.

В разделе <<Средства и порядок испытаний>> указаны средства, с помощью которых проводятся испытания и указан порядок испытаний.

В разделе <<Методы испытаний>> содержится описание действий, требуемых для испытания программного продукта и приведен результат испытаний.

Настоящая Программа и методика испытаний удовлетворяет требованиям ГОСТ 19.301-78~\cite{espd301}.

\tableofcontents

\section{Объект испытаний}
\subsection{Наименование программы}
\paragraph{Наименование программы на русском языке}
Программная среда для записи математических лекций <<LexType>>.
\paragraph{Наименование программы на английском языке}
Software Environment for Mathematical Lectures Recording <<LexType>>.

\subsection{Область применения}
Программа предназначена для запись лекций в университетах и колледжах по математическим и техническим предметам. Соответственно, программа позволяет конспектировать в условиях, которые отличаются необходимостью частой записи математических формул и схематических рисунков. Также приложение может использоваться и в других сферах, не требующих возможности быстрой записи: подготовка курсовых и дипломных работ, написание научных статей, ведение рабочих записей.

\subsection{Обозначение испытуемой программы}
Краткое обозначение испытуемой программы -- <<LexType>>.

\section{Цель испытаний}
\subsection{Цель проведения испытаний}
Целью испытаний является проверка заявленных в Техническом задании требований к функциональности и надежности программного комплекса. Проведение испытаний обосновано в Техническом задании.

\section{Требования к программе}
\subsection{Требования к функциональным характеристикам}
Основное окно программы должно включать текстовый редактор, который должен обеспечивать ввод и обработку текстовой информации, включая поддержку символов Unicode. Пользователь должен иметь возможность передвигать окно текста для полного просмотра, перемещать каретку, путем клика мышкой по нужной позиции в тексте или нажиманием стрелок. Должна быть возможность выделять текст, путем перемещения мыши с зажатой левой кнопкой мыши или нажатием стрелок с зажатой клавишей Shift.

Для работы с файлами конспектов должна быть возможность открыть файл (в формате .tex или .lex), а также сохранить его, в тех же форматах. Для этих, а так же для других регулярных операций (т.е. всех, кроме настроек) должны быть предусмотрены сочетания клавиш для быстрого доступа.

Также текстовый редактор должен обладать подсветкой синтаксиса (подсветка команд, формул LaTeX и препроцессорного языка). Кроме этого, должна быть возможность использования сниппетов, т.е. подстрок, которые, при появлении перед курсором, автоматически заменяются на другую подстроку с определенным перемещением курсора. Сниппеты должны различаться на математические и регулярные, при этом первые должны срабатывать в математическом контексте, а вторые -- в любом другом. Пользователь должен иметь возможность настраивать сниппеты.

Программа должна обладать пунктом настроек, в котором пользователь может менять сниппеты, включать и выключать их, цвет расцветки. На уровне кода должна быть обеспечена возможность добавления новых настроек.

Векторный редактор должен позволять строить векторные изображения, используя примитивы (линии, фигуры, текст). На уровне кода должна присутствовать возможность добавления новых изображений. Должна быть возможность удалять построенные фигуры, а также использовать модификатор (кнопка Shift) для дополнительных функций (преобразование эллипса в окружность, привязка наклона прямой к кратным 15 градусам углам). Использование графического редактора должно быть возможно с использованием клавиатуры (для открытия и вставки, смены модификаторов и фигур). Текст должен вставляться в документ в текстовом виде с возможностью компиляции в PDF.

\subsection{Требования к надежности}
Приложение должно проверять входные данные для обеспечения корректной работы. В коде должны использоваться комманды assert для параметров функций, а так же тесты, в качестве дополнительной меры контроля правильности программы.

\section{Требования к программной документации}
\subsection{Состав программной документации}\label{subsection:documentation}
<<Программная среда для записи математических лекций>>. Техническое задание (ГОСТ 19.201-78~\cite{espd201})

<<Программная среда для записи математических лекций>>. Программа и методика испытаний (ГОСТ 19.301-78~\cite{espd301})

<<Программная среда для записи математических лекций>>. Пояснительная записка (ГОСТ 19.404-79~\cite{espd404})

<<Программная среда для записи математических лекций>>. Руководство оператора (ГОСТ 19.505-79~\cite{espd505})

<<Программная среда для записи математических лекций>>. Текст программы (ГОСТ 19.401-78~\cite{espd401})

\subsection{Специальные требования к программной документации}
Документы к программе должны быть выполненны в соответствии с ГОСТ 19.106-78~\cite{espd106} и ГОСТами к каждому виду документа (см. п.~\ref{subsection:documentation}).

Пояснительная записка должна быть загружена в систему <<Антиплагиат>> через LMS НИУ ВШЭ.

Документация и программа сдаются в электронном виде в формате .pdf или .docx в архиве формата .zip или .rar.

За один день до защиты комиссии все материалы курсового проекта:
\begin{enumerate}
\item Техническая документация;
\item Программный проект;
\item Исполняемый файл;
\item Отзыв руководителя;
\item Лист Антиплагиата.
\end{enumerate}
должны быть загружены одним или несколькими архивами в проект дисциплины <<Курсовой проект 2019-2020>> в личном кабинете информационной образовательной среде LMS (Learning Management System) НИУ ВШЭ.

\section{Средства и порядок испытаний}
\subsection{Технические и программные средства, используемые во время испытаний}
Во время испытаний приложения используется компьютер Dell Vostro 5471 со следующими характеристиками:

\begin{enumerate}
\item Процессор Intel Core i5-8250U (частота 1.8 ГГц);
\item Оперативная память 8 Гб;
\item Раздел, использующийся при тестировании для программы: 2 Гб свободного пространства, SSD;
\item Сенсорная панель, клавиатура и монитор встроены.
\end{enumerate}

Установлена операционная система Ubuntu 18.04 LTS.

\subsection{Порядок проведения испытаний}
\begin{enumerate}
\item Проверка документации;
\item Проверка интерфейса;
\item Проверка функционала.
\end{enumerate}

\section{Методы испытаний}
\subsection{Описание используемых методов испытаний для проверки программной документации}
Документация проверяется на соответствие составу (см. п.~\ref{subsection:documentation}). Также должно быть соответствие стандартам ГОСТ 19 ЕСПД, указанным в этоп пункте.

\subsection{Описание используемых методов испытаний для проверки требований к интерфейсу}
Приложение испытывается на операционной системе Ubuntu. При запуске программа выглядит следующим образом (\ref{window}):

\illustration[Окно программы]{img/window}[window]

Как видно, в программе действительно присутствует текстовый редактор. Так же видно панель управления, со всеми заявленными пунктами. Введем строку <<doc>> и нажмем Tab (\ref{doc}):

\illustration[Сниппет <<doc>>]{img/doc}[doc]

Применяется сниппет для вставки заготовки документа. При этом ключевые слова LaTeX (documentclass, begin, end), а так же фигурные скобки, выделяются специальным цветом, соответственно, подсветка синтаксиса действительно работает.

Нажмем сочетание клавиш Ctrl+D (Или выберем пункт Painter в меню). Откроется графический редактор (\ref{painter}):

\illustration[Векторный редактор изображений]{img/painter}[painter]

Графический редактор имеет вид панели для рисования с набором инструментов на верхней грани диалогового окна. Закроем это окно, снова нажав сочетание клавиш Ctrl+D, после чего в пункте меню Tools найдем настройки и откроем их (\ref{settings}):

\illustration[Настройки]{img/settings}[settings]

Обозрев настройки, можно видеть, что, действительно, заявленные пункты настроек присутствуют.

\subsection{Описание используемых методов испытаний для проверки требований к функциональности}

Наличие сниппетов и их работоспособность была уже испытана с помощью сниппета <<doc>>. Для проверки работоспособности контекста добавим в заготовку документа строку <<`doc`>> и нажмем Tab. Ничего не произойдет, так как данный сниппет привязан к контексту регулярного текста. Напишем текст <<for>> после пробела, при этом вставится квантор общности. Соответственно, сниппеты, их контексты и конечные позиции работают (\ref{for}).

\illustration[Математический контекст сниппета]{img/for}[for]

Приступим к тестированию редактора изображений. Перейдем на следующую строку, после чего нажмем Ctrl+D, выберем инструмент <<Rectange>> и вставим его в область рисования (\ref{rectangle}).

\illustration[Прямоугольник в редакторе]{img/rectangle}[rectangle]

Нажмем Ctrl+Z и удалим фигуру. Соответственно, удаление фигур работает. Теперь выберем инструмент (нажав кнопку L) <<Line>>, поставим модификатор двойной стрелки и, зажав Shift, нарисуем горизонтальную линию. Видно, что оба модификатора работают, в редакторе отобразилась горизонтальная двунаправленная стрелка (\ref{line}).

\illustration[Модификаторы фигур]{img/line}[line]

Нажмем Ctrl+D. нарисованное изображение вставится в окно текстового редактора в формате TikZ (\ref{tikz}).

\illustration[Изображение после вставки]{img/tikz}[tikz]

Сохраним файл в формате под названием <<file.lex>>. Нажмем F5, при этом произойдет компиляция программы. Запустим просмотрщик файлов, откроем сгенерированный файл <<file.pdf>>. Используем сочетание клавиш Ctrl++, чтобы увеличить видимое изображение (\ref{viewer}).

\illustration[Компиляция и просмотр]{img/viewer}[viewer]

Как видно, все заявленные функции работают, компиляция происходит, приложение позволяет редактировать математический текст и вставлять изображения.

\bibliography{espd,library}

\end{document}
